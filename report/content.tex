\begin{abstract}
\end{abstract}


\section{Random time change representation}
Consider a model of a system with states $A$ for closed and $B$ for open of an ion channel.
The dynamics of the system are modeled assuming that the dwell times in the two states are determined by independent exponential random variables with parameters $\alpha$ and $\beta$:

$$A\xleftrightharpoons[\beta]{\alpha}B$$

The probability that a closed channel opens in the next increment of time $\Delta s$ is assumed $\alpha\Delta s + o(\Delta s)$, while the probability that an open channel closes is assumed $\beta\Delta s + o(\Delta s)$.
This model can be described mathematically by a chemical master equation, that for the two states model is:

$$\begin{cases}\frac{d}{dt}p_{x_0}(A, t) = -\alpha p_{x_0}(A, t) + \beta p_{x_0}(B, t)\\\frac{d}{dt}p_{x_0}(B, t) = -\beta p_{x_0}(B, t) + \alpha p_{x_0}(A, t)\end{cases}$$

Where:

\begin{itemize}
	\item $p_{x_0}(x, t)$ is the probability of being in state $x\in \{A, B\}$ at time $t$ given the initial condition $x_0$.
	\item $x_0$ is the initial condition.
\end{itemize}

The chemical master equation is a linear ODE governing the dynamical behaviour of the probability distribution of the model and does not provide a stochastic representation for a particular realization of the process.
To reconstruct  a path-wise representation let:

\begin{itemize}
	\item $R_1(t)$ be the number of times $A\rightarrow B$ has taken place by time $t$.
	\item $R_2(t)$ be the number of times $B\rightarrow A$ has taken place by time $t$.
	\item $X_1(t)\in\{0,1\}$ be $1$ if the channel is closed at time $t$ and zero otherwise.
	\item $X_2(t)=1-X_1(t)$ be $1$ if the channel is open at time $t$ and zero otherwise.
	\item $X(t) = \begin{pmatrix}X_1(t) & X_2(t)\end{pmatrix}^T$.
\end{itemize}

Now:

$$X(t) = X(0) + R_1(t)\begin{pmatrix}-1\\1\end{pmatrix} + R_2(t)\begin{pmatrix}1\\-1\end{pmatrix}$$

The counting processes $R_1$ and $R_2$ are represented as unit-rate Poisson processes.
A unit-rate Poisson process can be constructed considering:

\begin{itemize}
	\item $\{e_i\}_{i=1}^{\infty}$ independent exponential random variables with a parameter of $1$.
	\item $\tau=e_1, \tau_2 = \tau_1+e_2, \dots, t_n = t_{n-1} + e_n$.
\end{itemize}

The associated unit-rate Poisson processes $Y(s)$ is the counting process determined by the number of points $\{\tau_i\}_{i=1}^{\infty}$ that come before $s \ge 0$.
Let $\lambda:[0, \infty[\rightarrow\mathbb{R}_{\ge 0}$ be the rate of movement along the time axis, then the number of points observed by time $s$ is:

$$Y\biggl(\int_0^{s}\lambda(r)dr\biggr)$$

From the basic properties of exponential random variables, whenever $\lambda(s)>0$, the probability of seeing a jump within the next small increment of time $\Delta s$ is:

$$P\biggl(Y\biggl(\int_{0}^{s+\Delta s}\lambda(r)dr\biggr)-Y\biggl(\int_{0}^{s}\lambda(r)dr\biggr)\ge 1\biggr)\sim \lambda(s)\Delta s$$

Thus the propensity for seeing another jump is $\lambda(s)$.
Noting that $\forall s, X_1(s) + X_2(s)=1$, the propensity for reactions $1$ and $2$ are:

$$\lambda_1(X(s))=\alpha X_1(s), \quad\lambda_2(X(s)) = \beta X_2(s)$$

Combining all of the above $R_1$ and $R_2$ can be represented as:

$$R_1(t) = Y_1\biggl(\int_o^t\alpha X_1(s)ds\biggr), \quad Y_2\biggl(int_0^t\beta X_2(s)ds\biggr)$$

So a pathways representation for the stochastic model can be obtained:

\begin{align*}
	X(t) = X_0 &+ Y_1\biggl(\int_0^t \alpha X_1(s)ds\biggr)\begin{pmatrix}-1\\1\end{pmatrix} +\\
						 &+ Y_2\biggl(\int_o^t\beta X_2(s)ds\biggr)\begin{pmatrix}1\\-1\end{pmatrix}
\end{align*}

Where $Y_1$ and $Y_2$ are independent, unit-rate Poisson processes.
Suppose now that $X_1(0) + X_2(0) = N\ge 1$.
Now the model is focusing on the number of open and closed ion channels out of a total of $N$.
Suppose that the propensity at which ion channels are opening can be modelled as:

$$\lambda_1(t, X(t)) = \alpha(t)X_1(t)$$

And the rate at which they close:

$$\lambda_2(t, X(t)) = \beta(t)X_2(t)$$

Where $\alpha(t)$ and $\beta(t)$ are non-negative functions of time, probably depending on voltage.
Suppose that for each $i\in\{1, 2\}$, the conditional probability of seeing the counting process $R_1$ increase in the interval $[t, t+h[$ is $\lambda_1(t, X(t))h+o(h)$.
The expression is now:

\begin{align*}
	X(t) = X_0 &+ Y_1\biggl(\int_0^t\alpha(s)X_1(s)ds\biggr)\begin{pmatrix}-1\\1\end{pmatrix} +\\
						 &+Y_2\biggl(\int_0^t\beta(s)X_2(s)ds\biggr)\begin{pmatrix}1\\-1\end{pmatrix}
\end{align*}




% \begin{figure}[hbt!]
% \centering
% \label{volcano}
% \end{figure}
