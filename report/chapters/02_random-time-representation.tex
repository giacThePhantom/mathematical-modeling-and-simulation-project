\section{Stochastic representations}

	\subsection{Random time change representation}
	Consider a model of a system with states $A$ for closed and $B$ for open of an ion channel.
	The dynamics of the system are modelled assuming that the dwell times in the two states are determined by independent exponential random variables with parameters $\alpha$ and $\beta$:

	$$A\xleftrightharpoons[\beta]{\alpha}B$$

	The probability that a closed channel opens in the next increment of time $\Delta s$ is assumed $\alpha\Delta s + o(\Delta s)$, while the probability that an open channel closes is assumed $\beta\Delta s + o(\Delta s)$.
	This model can be described mathematically by a chemical master equation, that for the two states model is:

	$$\begin{cases}\frac{d}{dt}p_{x_0}(A, t) = -\alpha p_{x_0}(A, t) + \beta p_{x_0}(B, t)\\\frac{d}{dt}p_{x_0}(B, t) = -\beta p_{x_0}(B, t) + \alpha p_{x_0}(A, t)\end{cases}$$

	Where:

	\begin{itemize}
		\item $p_{x_0}(x, t)$ is the probability of being in state $x\in \{A, B\}$ at time $t$ given the initial condition $x_0$.
		\item $x_0$ is the initial condition.
	\end{itemize}

	The chemical master equation is a linear ODE governing the dynamical behaviour of the probability distribution of the model and does not provide a stochastic representation for a particular realization of the process.
	To reconstruct  a path-wise representation let:

	\begin{itemize}
		\item $R_1(t)$ be the number of times $A\rightarrow B$ has taken place by time $t$.
		\item $R_2(t)$ be the number of times $B\rightarrow A$ has taken place by time $t$.
		\item $X_1(t)\in\{0,1\}$ be $1$ if the channel is closed at time $t$ and zero otherwise.
		\item $X_2(t)=1-X_1(t)$ be $1$ if the channel is open at time $t$ and zero otherwise.
		\item $X(t) = \begin{pmatrix}X_1(t) & X_2(t)\end{pmatrix}^T$.
	\end{itemize}

	Now:

	$$X(t) = X(0) + R_1(t)\begin{pmatrix}-1\\1\end{pmatrix} + R_2(t)\begin{pmatrix}1\\-1\end{pmatrix}$$

	The counting processes $R_1$ and $R_2$ are represented as unit-rate Poisson processes.
	A unit-rate Poisson process can be constructed considering:

	\begin{itemize}
		\item $\{e_i\}_{i=1}^{\infty}$ independent exponential random variables with a parameter of $1$.
		\item $\tau=e_1, \tau_2 = \tau_1+e_2, \dots, \tau_n = \tau_{n-1} + e_n$.
	\end{itemize}

	The associated unit-rate Poisson processes $Y(s)$ is the counting process determined by the number of points $\{\tau_i\}_{i=1}^{\infty}$ that come before $s \ge 0$.
	Let $\lambda:[0, \infty[\rightarrow\mathbb{R}_{\ge 0}$ be the rate of movement along the time axis, then the number of points observed by time $s$ is:

	$$Y\biggl(\int_0^{s}\lambda(r)dr\biggr)$$

	From the basic properties of exponential random variables, whenever $\lambda(s)>0$, the probability of seeing a jump within the next small increment of time $\Delta s$ is:

	$$P\biggl(Y\biggl(\int_{0}^{s+\Delta s}\lambda(r)dr\biggr)-Y\biggl(\int_{0}^{s}\lambda(r)dr\biggr)\ge 1\biggr)\sim \lambda(s)\Delta s$$

	Thus the propensity for seeing another jump is $\lambda(s)$.
	Noting that $\forall s, X_1(s) + X_2(s)=1$, the propensity for reactions $1$ and $2$ are:

	$$\lambda_1(X(s))=\alpha X_1(s), \quad\lambda_2(X(s)) = \beta X_2(s)$$

	Combining all of the above $R_1$ and $R_2$ can be represented as:

	$$R_1(t) = Y_1\biggl(\int_o^t\alpha X_1(s)ds\biggr), \quad Y_2\biggl(\int_0^t\beta X_2(s)ds\biggr)$$

	So a pathways representation for the stochastic model can be obtained:

	\begin{align*}
		X(t) = X_0 &+ Y_1\biggl(\int_0^t \alpha X_1(s)ds\biggr)\begin{pmatrix}-1\\1\end{pmatrix} +\\
							 &+ Y_2\biggl(\int_o^t\beta X_2(s)ds\biggr)\begin{pmatrix}1\\-1\end{pmatrix}
	\end{align*}

	Where $Y_1$ and $Y_2$ are independent, unit-rate Poisson processes.
	Suppose now that $X_1(0) + X_2(0) = N\ge 1$.
	The model is focusing on the number of open and closed ion channels out of a total of $N$.
	Suppose that the propensity at which ion channels are opening can be modelled as:

	$$\lambda_1(t, X(t)) = \alpha(t)X_1(t)$$

	And the rate at which they close:

	$$\lambda_2(t, X(t)) = \beta(t)X_2(t)$$

	Where $\alpha(t)$ and $\beta(t)$ are non-negative functions of time, probably depending on voltage.
	Suppose that for each $i\in\{1, 2\}$, the conditional probability of seeing the counting process $R_1$ increase in the interval $[t, t+h[$ is $\lambda_1(t, X(t))h+o(h)$.
	The expression is now:

	\begin{align*}
		X(t) = X_0 &+ Y_1\biggl(\int_0^t\alpha(s)X_1(s)ds\biggr)\begin{pmatrix}-1\\1\end{pmatrix} +\\
							 &+Y_2\biggl(\int_0^t\beta(s)X_2(s)ds\biggr)\begin{pmatrix}1\\-1\end{pmatrix}
	\end{align*}

	A jumping model consisting of $d$ chemical constituents (ion channel states) undergoing transitions determined via $M>0$ different reaction channels.
	Moreover suppose that $X_i(t)$ determines the value of the ith constituent at time $t$, so that $X(t)\in\mathbb{Z}^{d}$, that the propensity function for the kth reaction is $\lambda_k(t, X(t))$ and that if the kth reaction channel takes place at time $t$, then the system is updated according to the addition of the reaction vector $\zeta_k\in\mathbb{Z}^d$:

	$$X(t) = X(t-)+\zeta_k$$

	The path-wise stochastic representation for the model is:

	$$X(t) = X_0 + \sum\limits_kY_k\biggl(\int_0^t\lambda_k(s, X(s))ds\biggr)\zeta_k$$

	Where $Y_k$ are independent unit-rate Poisson processes.
	The chemical master equation is:

	\begin{align*}
		\frac{d}{dt} &= \sum\limits_{k=1}^MP_{X_0}(x-\zeta_k, t)\lambda(t, x-\zeta_k)+\\
								 &-P_{X_0}(x, t)\sum\limits_{k=1}^M\lambda_k(t, x)
	\end{align*}

	Where $P_{X_0}(x, t)$ is the probability of being in state $x\in\mathbb{Z}^d_{\ge 0}$ at time $t\ge 0$ given an initial condition of $X_0$.\\

	When $X$ represents the randomly fluctuating state of an ion channel in a single compartment conductance based neuronal model, the membrane potential $V\in\mathbb{R}$ is added as an additional dynamical variable.
	The voltage is considered to evolve deterministically, conditional on the states of the ion channels.
	Suppose that there is a single ion channel type with state variable $X$, then the path-wise representation is supplemented with the solution of Kirchoff's current conservation law:

	$$C\frac{dV}{dt} = I_{app}(t)-I_V(V(t))-\biggl(\sum\limits_{i=1}^dg_i^oX_i(t)\biggr)(V(t)-V_X)$$

	Where:

	\begin{itemize}
		\item $g_i^o$ is the conductance of an individual channel in the ith state.
		\item The sum gives the total conductance associated with the channel represented by the vector $X$.
		\item The reversal potential is $V_X$.
		\item $I_V(V)$ captures deterministic voltage-dependent currents due to other channels.
		\item $I_{app}$ is a time-varying, deterministic applied current.
	\end{itemize}

	The propensity function will be a function of the voltage, so $\lambda_k(s, X(s))$ will be replaced with $\lambda_k(V(s), X(s))$.
	If there are a finite number of types of channels, the vector $X\in\mathbb{Z}^d$ represents the aggregated channel state.

		\subsubsection{Simulation of the representation}
		This representation implies a simulation strategy in which each point of the Poisson processes $Y_k$ denoted $\tau_n$ is generated sequentially as needed.
		So the time until the next reaction that occurs past time $T$ is:

		$$\Delta = \min\limits_k\biggl\{\Delta_k:\int_0^{T+\Delta_k}\lambda_k(s, X(s))ds = \tau_T^k\biggr\}$$

		Where $\tau_T^k$ is the first point associated with $Y_k$ coming after $\int_0^T\lambda_k(s, X(s))ds$:

		\begin{align*}
			\tau_T^k = \inf\biggl\{&r>\int_0^T\lambda_k(s, X(s))ds:\\
														 &Y_k(r)-Y_k\biggl(\int_0^T\lambda_k(s, X(s))ds\biggr)=1\biggr\}
		\end{align*}

		The reaction that took place is the index at which the minimum is achieved.
		This allows to write the algorithm \ref{algo:random-time-change-sim}.

		\begin{algorithm}[hbt!]
\DontPrintSemicolon
\SetKwComment{comment}{$\%$}{}
\SetKw{Int}{int}
\SetKw{To}{to}
\SetKw{Return}{return}
\SetKw{Not}{not}
\SetKwData{Item}{item}
\SetKwFunction{Min}{min}
\SetKwFunction{TitleFunction}{RandomTimeChangeSimulation}

\caption{\protect\TitleFunction{}}
$X$ = Initial number of molecules of each species\;
$V$ = Initial voltage\;
$t = 0$\;
$T = 0$\;
\ForEach{$k$}{
	$\tau_k = 0$\;
	$T_k = 0$\;
}

$\{r_k\} = [\cdots]$\;


\For{$i = 0$ \To M}{
	$r_k[i] = norm(0,1)$\;
	$\tau_k = \ln\biggl(\frac{1}{r_k}\biggr)$\;
}
\While{$T<T_{\max}$}{
	$\Delta = 0$\;
	\While{$\int_{t}^{t+\Delta}\lambda_k(V(s), X(s))ds\neq \tau_k-T_k$}{
		Integrate forward in time: $C\frac{dV}{dt} = I_{app}(t)-I_V(V(t))-\biggl(\sum\limits_{i=1}^dg_i^oX_i(t)\biggr)(V(t)-V_X)$\;
		$\Delta += \Delta t$\;

	}

	$\mu = k$ such that $\int_{t}^{t+\Delta}\lambda_k(V(s), X(s))ds \tau_k-T_k$\;

	\ForEach{$k$}{
		$T_k = T_k + \int_{t}^{t+\Delta}\lambda_k(V(s), X(s))ds$\;
	}

	$t += \Delta$\;

	$X \leftarrow X + \zeta_\mu$\;

	$r = norm(0,1)$\;

	$\tau_\mu += \ln\biggl(\frac{1}{r}\biggr)$\;
	$T = T+ t$\;
}

\Return to the while or quit


\label{algo:random-time-change-sim}
\end{algorithm}


		In this algorithm $T_k$ denotes the value of the integrated intensity function $\int_0^t\lambda_k(s, X(s))ds$ and $\tau_k$ the first point associated with $Y_k$ located after $T_k$.
		All random number are assumed to be independent.
		With a probability of one, the index $\mu$ is unique at each step.
		This algorithm also relies on being able to compute a hitting time for each of the $T_k(t) = \int_0^t\lambda_k(s, X(s))ds$ exactly, which is in general not possible, but it will be sufficient with reliable integration software.
		If the equations for the voltage or the intensity functions can be analytically solved, such numerical integration is unnecessary and efficiencies can be gained.
