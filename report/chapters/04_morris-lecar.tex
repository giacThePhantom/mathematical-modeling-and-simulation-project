\section{Morris-Lecar}
The Morris-Lecar system is a concrete illustration of the exact stochastic simulation algorithms.
It was developed as a model for oscillation observed in barnacle muscle fibers.
The determinist equations constitute a planar model for the evolution of the membrane potential $v(t)$ and the fraction of potassium gates $n\in[0,1]$ that are in the open state.
In addition to this there is a depolarizing current gated by a rapidly equilibrating variable $m$, the calcium conductance which is treated now as a fast, deterministic variable as in the standard fast/slow decomposition or the planar Morris-Lecar model.
The mean field equation for this model are:

\begin{align*}
	\frac{dv}{dt} = f(v, n) =& \frac{1}{C}(I_{app}-g_{Ca}m_{\infty}(v)(v-v_{Ca})+\\
													 &-g_L(v-v_L)-g_Kn(v-v_K))
\end{align*}

$$\frac{dn}{dt} = g(v, n) = \alpha(v)(1-n)-\beta(v)n = \frac{n_{\infty}(v)-n}{\tau(v)}$$

The kinetics of the potassium channel can be specified by the instantaneous time constant $\tau$, the asymptotic target $n_{\infty}$ or by the per capita transition rates $\alpha$ and $\beta$.
Moreover:

$$m_{\infty} = \frac{1}{2}\biggl(1+\tanh\biggl(\frac{v-v_a}{v_b}\biggr)\biggr)$$

$$\alpha(v) = \frac{\phi\cosh\bigl(\frac{\epsilon}{2}\bigr)}{1+e^{2\epsilon}}$$

$$\beta(v) = \frac{\phi\cosh\bigl(\frac{\epsilon}{2}\bigr)}{1+e^{-2\epsilon}}$$

$$n_\infty(v) = \frac{\alpha(v)}{\alpha(v)+\beta(v)} = \frac{1+\tanh\epsilon}{2}$$

$$\tau(v) = \frac{1}{\alpha(v)+\beta(v)} = \frac{1}{\phi\cosh\frac{\epsilon}{2}}$$

Where $\epsilon = \frac{v-v_c}{v_d}$ and the parameter will have values:

\begin{multicols}{3}
	\begin{itemize}
		\item $v_K = -84$.
		\item $v_L = -60$.
		\item $v_{Ca} = 120$.
		\item $I_{app} = 100$.
		\item $g_K = 8$.
		\item $g_L = 2$.
		\item $C = 20$.
		\item $v_a = -1.2$.
		\item $v_b = 18$.
		\item $v_c = 2$.
		\item $v_d = 30$.
		\item $\phi = 0.04$.
		\item $g_{Ca} = 4.4$.
	\end{itemize}
\end{multicols}

For which the deterministic system has a stable limit cycle.
For smaller values of the applied currents the system has a stable fixed point that loses stability through a subcritical Hopf bifurcation as $I_{app}$ increases.
A finite number of potassium channels $N_{tot}$ is introduced and the number of open channels are treated as a discrete random process.
Each potassium channel switches between the closed or open state independently of the other with voltage-dependent per capita transition rates $\alpha$ and $\beta$.
The entire population conductance ranges from $0$ to $g_K^o = \frac{g_K}{N_{tot}}$.
In this simulation $N_{tot} = 40$.
The random variables will be the voltage and the number of open potassium channel.
In the random time change representation the opening and closing of the potassium channels are driven by two independent unit rate Poisson processes $Y_{open}(t)$ and $Y_{close}(t)$.
The evolutions  of $V$ and $N$ are linked.
Whenever $N=n$, the evolution of $V$ obeys a deterministic differential equation:

$$\frac{dV}{dt}\biggr\vert_{N=n} = f(V, n)$$

$N$ evolves as a jump process: $N(t)$ is a piece-wise constant, with transitions occurring with intensities dependent on $V$.
Whenever $V = v$:

$$N\rightarrow N + 1\text{ net rate } \alpha(v)(N_{tot}-N)$$

$$N\rightarrow N - 1\text{ net rate } \beta(v)N$$

Now representing the state space for $N$ graphically:

\begin{align*}
&0\xleftrightharpoons[\beta]{\alpha N_{tot}} 1 \xleftrightharpoons[2\beta]{\alpha(N_{tot}-1)}2\xleftrightharpoons[3\beta]{\alpha(N_{tot}-2)} \dots \xleftrightharpoons[k\beta]{\alpha(N_{tot}-k+1)}k&\\
&\scriptstyle{\alpha(N_{tot}-k)}\downharpoonleft\upharpoonright&\scriptstyle{(k+1)\beta}\\
& N_{tot}\xleftrightharpoons[N_{tot}\beta]{\alpha}(N_{tot}-1)\xleftrightharpoons[\beta(N_{tot}-1)]{2\alpha}\dots\xleftrightharpoons[(k+2)\beta]{\alpha(N_{tot}-k-1)}(k+1)&
\end{align*}

The nodes represent possible states for process $N$ an the transition intensities located above and below the arrows.
Adopting the random time change representation the stochastic Morris-Lecar system is written as:

\begin{align*}
	\frac{dV}{dt} &=& f(V(t), N(t)) = \\
								&=& \frac{1}{C}(I_{app}-g_{Ca}m_\infty(V(t))(V(t)-V_{Ca})-g_L(V-V_L) +\\
								& & - g_K^oN(t)(V(t)-V_K))
\end{align*}

\begin{align*}
	N(t) = N(0) -&Y_{close}\biggl(\int_0^t\beta(V(s))N(s)ds\biggr) + \\
							 &Y_{open}\biggl(\int_0^t\alpha(V(s))(N_{tot}-N(s))ds\biggr)
\end{align*}
