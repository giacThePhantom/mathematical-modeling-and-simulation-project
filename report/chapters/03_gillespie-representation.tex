	\subsection{Gillespie representation}
	The Gillespie representation is an alternative representation for the stochastic process built before.
	Let:

	\begin{itemize}
		\item $Y$ be a unit rate Poisson process.
		\item $\{\epsilon_i, i = 0, 1, 2, \dots\}$ be independent, uniform $(0,1)$ random variables independent of $Y$.
	\end{itemize}

	Set:

	$$\lambda_0(V(s), X(s)) = \sum\limits_{k=1}^M\lambda_k(V(s), X(s))$$

	$$q_0 = 0$$

	And $\forall k\in\{1, \dots, M\}$:

	$$q_k(s) = \lambda_0(V(s), X(s))^{-1}\sum\limits_{l=1}^k\lambda_l(V(s), X(s))$$

	Where $X$ and $Y$ satisfy:

	$$R_0(t) = Y\biggl(\int_o^t\lambda_0(V(s), X(s))ds\biggr)$$

	\begin{align*}
		X(t) =& X(0) + \\
					&\sum\limits_{k=1}^M\zeta_k\int_0^t1\{\epsilon_{r_0(s-)}\in[q_{k-1}(s-), q_k(s-)[\}dR_0(s)
	\end{align*}

	$$C\frac{dV}{dt} = i_{app}(t)-I_V(V(t)) - \biggl(\sum\limits_{i=1}^dg_o^oX_i(t)\biggr)(V(t)-V_X)$$

	The stochastic process $(X, V)$ descried in the equation above is a Markov process equivalent to the one described in the previous section.
	To understand it note that $R_0(t)$ determines the holding time in each state and the middle equation determines the embedded discrete time Markov chain or skeletal chain in the usual manner.
	This is the representation for the Gillespie algorithm, with time dependent propensity functions.
	This representation is analogous to the PDMP formalism with $\lambda_0$ the rate function that determines the time of the next jump and the middle equation the transitions.

		\subsubsection{Simulation of the representation}
		The simulation is analogous of using Gillespie's algorithm in the time-homogeneous case.
		The algorithm is \ref{algo:gillespie} and all number are assumed to be independent.
		In our implementation in Matlab the chosen solver for the ODE was \emph{ode23}, the low order method \cite{ode23}.

		
\begin{algorithm}[hbt!]
\DontPrintSemicolon
\SetKwComment{comment}{$\%$}{}
\SetKw{Int}{int}
\SetKw{To}{to}
\SetKw{Return}{return}
\SetKw{Not}{not}
\SetKwData{Item}{item}
\SetKwFunction{Min}{min}
\SetKwFunction{TitleFunction}{GillespieRepresentation}

\caption{\protect\TitleFunction{}}
$X$ = Initial number of molecules of each species\;
$V$ = Initial voltage\;
$t = 0$\;

$r = norm(0,1)$\;

\While{$\int_{t}^{t+\Delta}\lambda_0(V(s), X(s))ds\neq \ln\biggl(\frac{1}{r}\biggr)$}{
	Integrate forward in time: $C\frac{dV}{dt} = I_{app}(t)-I_V(V(t))-\biggl(\sum\limits_{i=1}^dg_i^oX_i(t)\biggr)(V(t)-V_X)$\;
	$\Delta += \Delta t$\;

}

$\epsilon = norm(0,1)$\;
$\Delta_1 = 0$\;
$k_1 = 0$\:
\ForEach{$k$}{
	\If{$\epsilon \in[q_{k-1}((t+\Delta)-), q_k((t+\Delta)-)]$}{
		$\Delta_1 = \Delta$\;
		$k_1 = k$\;
	}
}

$t += \Delta_i$\;

$X \leftarrow X + \zeta_{k_1}$\;

\Return to the while or quit


\label{algo:gillespie}
\end{algorithm}


		This algorithm also relies on being able to compute a hitting time.
