\section{Models with more than one channel type}
Now the original Morris-Lecar model is considered, where there is a three-dimensional phase space, taking both the calcium and potassium channel to be discrete.

	\subsection{Random time change representation}
	In the original treatment of voltage oscillations  in barnacle muscle fibres the calcium gating variable $m$ is included as a dynamical variable, sot the total system is describes with the following deterministic equations:


	\begin{align*}
		\frac{dv}{dt} = F(v, n, m) =&\frac{1}{C}(I_{app}-g_L(v-v_L)+\\
																&-g_{Ca}m(v-v_Ca)-g_Kn(v-v_K))
	\end{align*}

	\begin{align*}
		\frac{dn}{dt} &= G(v, n, m) = \alpha_n(v)(1-n)-\beta_n(v)n=\\
									&=\frac{n_{\infty}(v)-n}{\tau_n(v)}
	\end{align*}

	\begin{align*}
		\frac{dm}{dt} &= H(v, n, m) = \alpha_m(v)(1-m)-\beta_m(v)m=\\
									&=\frac{m_\infty(v)-m}{\tau_m(v)}
	\end{align*}

	The number of calcium gates evolves according to this equation instead of being set to its asymptotic value $m_\infty = \frac{\alpha_m}{\alpha_m+\beta_m}$.
	The planar form of the equation is obtained by observing that $m$ approaches equilibrium faster than $n$ and $v$, so using standard arguments from singular perturbation theory this system can be brought into the planar model by replacing $F$ and $G$ with: $f(v, n) = F(v, n, m_\infty(v))$ and $g(v, n) = G(v, n, m_\infty(v))$.
	Now for the $3D$ model $\epsilon_m = \frac{v-v_a}{v_b}$ has to be introduced in addition to $\epsilon_n = \frac{v-v_c}{v_d}$.
	It can be noted how $\epsilon_x$ represents where the voltage falls along the activation curve for channel type $x$, relative to its half-activation point ($v_a$ for calcium and $v_c$ for potassium )and its slope (reciprocals of $v_b$ for calcium and $v_d$ for potassium).
	The per capita opening and closing rates for each channel type are then:

	$$\alpha_m(v) = \frac{\phi_m\cosh\frac{\epsilon_m}{2}}{1+e^{2\epsilon_m}}\quad\beta_m(v) = \frac{\phi_m\cosh\frac{\epsilon_m}{2}}{1+e^{-2\epsilon_m}}$$

	$$\alpha_n(v) = \frac{\phi_n\cosh\frac{\epsilon_n}{2}}{1+e^{2\epsilon_n}}\quad\beta_n(v) = \frac{\phi_n\cosh\frac{\epsilon_n}{2}}{1+e^{-2\epsilon_n}}$$

	With parameters:

	\begin{multicols}{3}
		\begin{itemize}
			\item $v_a = -1.2$.
			\item $v_b = 18$.
			\item $v_c = 2$.
			\item $v_d = 30$.
			\item $\phi_m = 0.4$.
			\item $\phi_n = 0.04$
		\end{itemize}
	\end{multicols}

	The asymptotic open probabilities are given by $m_\infty$ and $n_\infty$ and the time constants $\tau_m$ and $\tau_n$, which satisfy the relations:

	$$m_\infty(v) = \frac{\alpha_m(v)}{\alpha_m(v) + \beta_m(v)} = \frac{1+\tanh\epsilon_m}{2}$$

	$$n_\infty(v) = \frac{\alpha_n(v)}{\alpha_n(v) + \beta_n(v)} = \frac{1+\tanh\epsilon_n}{2}$$

	$$\tau_m(v) = \frac{1}{\phi\cosh\frac{\epsilon_m}{2}}$$

	$$\tau_n(v) = \frac{1}{\phi\cosh\frac{\epsilon_n}{2}}$$

	Assuming a population of $M_{tot}$ calcium and $N_{tot}$ potassium gates a stochastic hybrid system is obtained, with a continuous variable $V(t)$ and two discrete ones $M(t)$ and $N(t)$.
	The voltage evolves according to the sum of the applied, leak, calcium and potassium currents

	\begin{align*}
		\frac{dV}{dt} &=&  F(V(t), N(t), M(t)) = \\
									&=& \frac{1}{C}\biggl(I_{app} - g_L(V(t)-v_L)-g_{Ca}\frac{M(t)}{M_{tot}}(V(t)-v_{Ca}) +\\
									& &- g_K\frac{N(t)}{N_{tot}}(V(t)-v_K)\biggr)
	\end{align*}

	The number of open gates change only by unit increases and decreases remaining constant between such changes.
	So the channel states evolve according to:

	\begin{align*}
		M(t) =&M(0) - Y_{close}^M\biggl(\int_0^t\beta_m(V(s))M(s)ds\biggr)+\\
					&+Y_{open}^M\biggl(\int_0^t\alpha_m(V(s))(M_{tot}-M(s))ds\biggr)
	\end{align*}

	\begin{align*}
		N(t) =&N(0) - Y_{close}^N\biggl(\int_0^t\beta_n(V(s))N(s)ds\biggr)+\\
					&+Y_{open}^N\biggl(\int_0^t\alpha_n(V(s))(N_{tot}-N(s))ds\biggr)
	\end{align*}
