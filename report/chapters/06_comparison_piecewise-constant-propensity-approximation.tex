\section{Comparison of the Random Time Change and Gillespie representations}
We compared the two representations running simulations that used different numbers $S_{i} = M_{tot} = N_{tot}$ with:

$$S = [1, 10, 50, 100]$$

The output of the simulations was the time evolution of the voltage $V$, the number of open potassium channels $N$ and the number of open calcium channels $M$, for each type of representation.

The way of comparing the representations involved generating heatmaps having 1000 binned values of the voltage as the $x$ axis and $k + 1 = M_tot + 1 = N_tot  + 1$ values on the $y$ axis.

Each cell of the heatmap represented the number of times a certain value of number of channels of a certain type appeared for that interval of binned voltage, summed over the other channel type.

The images generated by the aforementioned process are available in \ref{appendix}. They are divided into four $2x2$ matrices corresponding different numbers of $k = M_{tot} = N_{tot}$, where the rows represent the type of simulation and the columns represent the type of channel. The matrices are sorted in ascending order to highlight how the two representations start to look increasingly similar as the number of channels increases. This holds true for both the channel types.
