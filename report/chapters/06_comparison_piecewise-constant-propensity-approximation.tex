\section{Comparison of the exact algorithm with a piecewise constant propensity approximation}
The most common implementation for hybrid ion channel models is an approximate method in which the per capita reaction propensities are held fixed between channel events: in particular in algorithm \ref{algo:random-time-change-sim}, $\int_t^{t+\Delta k}\lambda_k(V(s), X(s))ds$ is replaced with $\Delta_k\lambda_k(V(t), X(t))$, leaving the remainder unchanged.
In this type of algorithms the sequence of channel states jumps is generated using the propensity immediately following the most recent jump rather than taking into account the time dependence of the reaction propensities due to the changing voltage.
This is analogous to the forward Euler method for the numerical solution of ordinary differential equations.
The solution of a stochastic differential equation with a given initial condition is a map from the sample space $\Omega$ to a space of trajectories.
In this context $\Omega$ is one independent unit rate Poisson process per reaction channel.
For the planar model a point in $\Omega$ amounts to fixing two Poisson process, $Y_{open}$ and $Y_{closed}$ to drive the transitions of the potassium channels.
For the full 3D model there are four processes:

\begin{multicols}{2}
	\begin{itemize}
		\item $Y_1 \equiv Y_{Ca, open}$.
		\item $Y_2 \equiv Y_{Ca, closed}$.
		\item $Y_3 \equiv Y_{K, open}$.
		\item $Y_4 \equiv Y_{K, closed}$.
	\end{itemize}
\end{multicols}

The exact algorithm provides a numerical solution of the map from $\{Y_k\}_{k=1}^4\in\Omega$ and initial conditions $(M_0, N_0, V_0)$ to the trajectory $(M(t), N(t), V(t))$.
The approximate piecewise algorithm gives a map from the same domain to a different trajectory $(\tilde{M}(t), \tilde{N}(t), \tilde{V}(t)$.
To make a pathways comparison the initial condition and the four Poisson processes are fixed and the resulting trajectories are compared.
Both algorithms produce a sequence of noise-dependent voltage spike with similar firing rates.
The two trajectories remain close together initially and the timing for the first spike is similar for both.
Over time discrepancies between the trajectories accumulate: the timing of the second and third spikes is different and before ten spikes the spike trains have become uncorrelated.
Even though the trajectories diverge the two processes could still generate sample paths with similar time-dependent or stationary distributions: the two algorithms could still be close in a weak sense.
Given $M_{tot}$ and $N_{tot}$ the density for the hybrid Markov process can be written as:

$$\rho_{m,n}(v,t) = \frac{1}{dv}Pr\{M(t) = m, N(t) = n, V\in[v, v+dv]\}$$

Obeying the master equation:

\begin{align*}
	\frac{\partial\rho_{m, n}(v, t)}{\partial t} =& - \frac{\partial(F(v, n, m)\rho_{m, n}(v, t))}{\partial v} +\\
																								&-(\alpha_m(v)(M_{tot}-m)+\beta_m(v)m+\\
																								&+\alpha_n(v)\cdot(N_{tot}-m)+\beta_n(v)n)\rho_{m, n}(v, t)+\\
																								&+(M_{tot}-m+1)\alpha_m(v)\rho_{m-1, n}(v, t)+\\
																								&+(m+1)\beta_m(v)\rho_{m+1, n}(v, t)+\\
																								&+(N_{tot}-n+1)\alpha_n(v)\rho_{m, n-1}(v,t)+\\
																								&+(n+1)\beta_n(v)\rho_{m, n+1}(v, t)
\end{align*}

With initial condition $\rho_{m,n}(v,0)\ge 0$ given by any integrable density such that $\in_{v\in\mathbb{R}}\sum\limits_{m,n}\rho_{m, n}(v, 0)dv \equiv 1$ and boundary conditions:$\rho\rightarrow 0$ as $|v|\rightarrow\infty$ and $\rho\equiv 0$ for either $m, n<0$ or $m> M_{tot}$ or $n>N_{tot}$.
The approximate algorithm does not generate a Markov process since the transition probabilities depend on the past rather than the present value of the voltage.
Because of this they do not satisfy the master equation, but it is plausible that they may have a unique stationary distributoin.

Looking the histograms in the $(v, n)$ plane with entries summed over $m$.
The algorithms were run with independent randomm number streams in the limit cycle regime ($I_{app} = 100$) fir $I_{max} \approx 200000$ time units, sampled every $10$.
Considering $N_{tot} = M_{tot} = k$, for $k<5$ the difference is obvious, while increasing $k$ the histograms become more and more similar.

Looking now at bar plots of the histograms with points projected on the voltage axis: entries summed over $m$ and $n$, the plots become increasingly similar the greater the $k$>

To quantify the similarity of the histogram the empirical $L_1$ difference between them has been computed.
For the full $(v, n, m)$ histograms and then for the collapsed voltage axis.
Let $\rho_{m,n}(v)$ and $\tilde{\rho}_{m, n}(v)$ denote the stationary distributions for the exact and approximate algorithms respectively.
To compare the two the $L_1$ distance between them is approximated:

$$d(\rho,\tilde{\rho}) = \int_{v_{\min}}^{v{\max}}\biggl(\sum\limits_{m=0}^{M_{tot}}\sum\limits_{n=0}^{N_{tot}}|\rho_{m, n}(v)-\tilde{\rho}_{m,n}(v)|\biggr)dv$$

Where $v_{min}$ and $v_{max}$ where chosen so that $F(v_{min}, n, m>0\land F(v_{max}, n, m)<0\forall m,n$.
Such value must exist since $F(v,n,m)$ is linear and monotonically decreasing for $v$ for any fixed pair of $(n,m)$.
For any exact simulation algorithm, once the voltage component falls between $v_{min}\le v \le v_{max}$ it remains in that interval for all time.
